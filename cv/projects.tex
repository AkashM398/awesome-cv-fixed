%-------------------------------------------------------------------------------
%	SECTION TITLE
%-------------------------------------------------------------------------------
\cvsection{Projects}

%-------------------------------------------------------------------------------
%	CONTENT
%-------------------------------------------------------------------------------
\begin{cventries}

%---------------------------------------------------------
  \cventry
    {Masters Thesis, Newcastle University} % Job title
    {An application to best support student education in semantics and best practices of object-oriented programming} % Organization
    {\href{https://github.com/mmacheerpuppy/bark}{Available on Github}} % Location
    {June. 2017 - September. 2017} % Date(s)
    {
      \begin{cvitems} % Description(s) of tasks/responsibilities
	\item {Provided a software solution which best facilitates student education of the Java programming language by the means of educating
the semantics of object-oriented programming as associable with the best practice of object-oriented programming.}
  \item {Involved in-depth qualitative research into establishing the best practices of object oriented programming as associated with understanding the kinds of meaningful language (semantics) when talking about concepts in object-oriented programming, and quantitative research to identify correlations between student failure in Java based courses and poor understanding of those semantics.}
  \item{Evaluated academic literature and statistics of various pedagogical strategies (learning styles) in synthesis to best teach those students the semantics and best practices of object oriented programming to justify choosing a single strategy from a disjunction.} 
  \item {Involved market research to existing solutions and how existing technologies can be integrated with the proposed solution.}
  \item{Using a waterfall software engineering process model developed a software solution utilizing a stack of loosely coupled Django applications organised into a layered architecture each designed using the single responsibility principle. The application stack allowed for teachers to serve course-content to students to teach the semantics and best practices of object oriented programming in Java based on student learning styles on a platform with user friendliness in mind.}
  \item{The application stack served course-content by generating internal virtual hosts on allocated ports. Users were dynamically proxy passed to the suitable service. Part of this solution involved the development of an application integrating Docker with Django to launch containerized NGINX web-servers to best scale, maintain, and deliver static-HTML course-content and additional features for end users, such as isolating a Java compilation environment.}
      \end{cvitems}
    }

  \cventry
    {Group Project, Newcastle University} % Job title
    {A railway-network timetable management client and optimised search engine} % Organization
    {\href{https://github.com/mmacheerpuppy/routes}{Available on Github}} % Location
    {December. 2016 - May. 2017} % Date(s)
    {
      \begin{cvitems} % Description(s) of tasks/responsibilities
  \item {For a mock railway service application (similar to 'trainline.com') I acted as group project team-leader, the focal point for group meetings and managing an agile development process. My key responsibilities included organising meetings, designing and modularizing the solution architecture, motivating feedback from team-members, escalating significant risk to the project as necessary, splitting the designed solution into manageable sets of objectives and milestones and ensuring they were met, and problem management (i.e.regularly supporting team members and troubleshooting their code in one-to-one sessions).}
  \item {I was solely responsible for the research, design, and implementation of a unique fastest time routing algorithm for the web-application, and presenting the implementation to examiners.}
  \item {Where a classical approach for modelling time-table information is a database of stations and trains stopping at each station, the designed implementation deviated from this approach, representing train-arrivals at stations by making train-times and stations conjunctive as event-based nodes. The topology of connected-nodes then identified which connecting train-times and stations were reachable from train-arrival events. Searches for fastest train times from A-B were then performable using a depth-first search through connected nodes.}
  \item {The routing solution was finally implemented as a module in the overall project using an object-oriented architecture written in PHP.}
      \end{cvitems}
    }

  \cventry
    {Personal Project} % Job title
    {Tournament matchmaking client} % Organization
    {\href{https://github.com/mmacheerpuppy/matchmaker}{Available on Github}} % Location
    {December. 2015} % Date(s)
    {
      \begin{cvitems} % Description(s) of tasks/responsibilities
  \item {Developed an application using a custom algorithm to rank, swap, and organize a database of active users on a gaming server into manageable sets (teams) of users of the same average rank apt to compete against each other in an online video game tournament.}
  \item {Produced and hosted that tournament using an NGINX RTMP server to broadcast to the larger involved community.}
      \end{cvitems}
    }

  \cventry
    {Task for Hewlett Packard Enterprise engineer} % Job title
    {Switch configuration client} % Organization
    {\href{https://github.com/mmacheerpuppy/switcharoo}{Available on Github}} % Location
    {December. 2014} % Date(s)
    {
      \begin{cvitems} % Description(s) of tasks/responsibilities
  \item {Delivered a network capable leveraged configuration management tool for deployment of a large number of stacked switches containing mixed switch models to Hewlett Packard Enterprise environments, requiring a degree of flexible options to create the appropriate stack mix.}
      \end{cvitems}
    }

%---------------------------------------------------------
\end{cventries}
